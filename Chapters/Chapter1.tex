% Indicate the main file. Must go at the beginning of the file.
% !TEX root = ../main.tex

%----------------------------------------------------------------------------------------
% Chapter 1
%----------------------------------------------------------------------------------------



\chapter{Introduction} % Main chapter title
\label{Chapter1} % For referencing the chapter elsewhere, use \ref{Chapter1} 

%----------------------------------------------------------------------------------------

% Define some commands to keep the formatting separated from the content
% Placing such commands in the preamble is a good idea.
\newcommand{\keyword}[1]{\textbf{#1}}
\newcommand{\tabhead}[1]{\textbf{#1}}
\newcommand{\code}[1]{\texttt{#1}}
\newcommand{\file}[1]{\texttt{\bfseries#1}}
\newcommand{\option}[1]{\texttt{\itshape#1}}

%----------------------------------------------------------------------------------------



%----------------------------------------------------------------------------------------

\section{Background and Significance}

Software development teams are responsible for integrating operational and development practices to improve the speed and quality of product delivery in this rapidly evolving field. To achieve this, they are adopting agile methodologies such as SCRUM. However, managing knowledge and making decisions in an efficient way is a challenge that these teams face to maintain their agility and effectiveness. This Bachelor's thesis, titled 'Defining a structure for decision logging and knowledge management in DevOps teams', aims to address the critical gap in systematic decision logging and knowledge management practices that could have a significant impact on the collaboration and productivity of these teams.

In software development, the careful documentation of decisions and the management of knowledge is crucial. The importance of this cannot be overstated. In dynamic and sometimes unpredictable contexts, understanding decision rationales and leveraging accumulated knowledge is critical. However, there is a lack of standardised practices for documenting and managing such information, tailored to the different roles within SCRUM teams. This thesis investigates the optimal formats for documenting critical data. It proposes a solution for effective maintenance and decision logs. The aim is to fill the gap in the literature that has been identified.
%----------------------------------------------------------------------------------------

\section{Research Questions}

\textbf{Three main questions guide the research:}
\begin{enumerate}[start=0,label={(\bfseries R\arabic*):}]
\item How to define the right format for documenting the right data for each team role in a SCRUM team?
\item How to implement a solution for maintaining and visualizing decision logs?
\item To what extent can collaboration and productivity be improved through these practices?
\end{enumerate}

%----------------------------------------------------------------------------------------

\section{Methodology and Approach}

This section provides an overview of the research methodology. Firstly, the chosen research method is discussed. This is followed by a description of the different stages of the research and the activities carried out at each stage.

The purpose of this study was to investigate the current state of art in SCRUM Teams of how the collaboration and the knowledge managment is working, where do Teams store their shared knowledge and how is it maintained, or is it maintained at all? As \cite{HumbleMolesky2011} is mentioning:

\begin{quote}
In most organizations, IT operations
consumes by far the majority of the IT budget. If you
can drive operating costs down by preventing and
removing bloat within systems created by projects,
you’d have more resources to focus on problem solving
and continuous improvement of IT services.
\end{quote}

Operations are quite expensive and consume a lot of money if handled ad-hoc and without the care. As \cite{Azad2023DevOps} is mentioning:

\begin{quote}
For software process integration the infrastructure play a crucial role
\end{quote}

Infrastructure is aswell as the knowledge a very important aspect within the Team, how can the infrastructure beeing managed and be deployed at the right place and within the right responsibilities.

The process model has been divided into three different stages.

\begin{enumerate}
    \item Problem identification
    \item Design and proposed solution
    \item Validation based on a Survey
\end{enumerate}


%----------------------------------------------------------------------------------------




