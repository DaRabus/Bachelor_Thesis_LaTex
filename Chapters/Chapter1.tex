% Indicate the main file. Must go at the beginning of the file.
% !TEX root = ../main.tex

%----------------------------------------------------------------------------------------
% Chapter 1
%----------------------------------------------------------------------------------------



\chapter{Introduction} % Main chapter title
\label{Chapter1} % For referencing the chapter elsewhere, use \ref{Chapter1} 

%----------------------------------------------------------------------------------------

% Define some commands to keep the formatting separated from the content
% Placing such commands in the preamble is a good idea.
\newcommand{\keyword}[1]{\textbf{#1}}
\newcommand{\tabhead}[1]{\textbf{#1}}
\newcommand{\code}[1]{\texttt{#1}}
\newcommand{\file}[1]{\texttt{\bfseries#1}}
\newcommand{\option}[1]{\texttt{\itshape#1}}

%----------------------------------------------------------------------------------------



%----------------------------------------------------------------------------------------

\section{Background and Significance}

Software Development teams are at the forefront of blending operational and development practices to improve the speed and quality of product delivery in this rapidly evolving space. By adopting agile methodologies such as SCRUM, these teams face the challenge of efficiently managing knowledge and decisions to maintain their agility and effectiveness. This Bachelor's thesis, entitled '\textbf{Defining a structure for decision logging and knowledge management in DevOps teams}', aims to address the critical gap in systematic decision logging and knowledge management practices that could have a significant impact on the collaboration and productivity of these teams.

The importance of carefully documenting decisions and managing knowledge in Software Development cannot be overstated. As these teams operate in dynamic and at times unpredictable contexts, understanding the reasons for decisions and leveraging accumulated knowledge is critical. However, despite its importance, there is a gap in standardized practices for the documentation and management of such information, tailored to the different roles within SCRUM teams. This thesis aims to fill this gap by investigating the optimal formats for the documentation of critical data and by proposing a solution for the effective maintenance and visualization of decision logs.

%----------------------------------------------------------------------------------------

\section{Research Questions}

\textbf{Three main questions guide the research:}
\begin{enumerate}
\item How to define the right format for documenting the right data for each team role in a SCRUM team?
\item How to implement a solution for maintaining and visualizing decision logs?
\item To what extent can collaboration and productivity be improved through these practices?
\end{enumerate}

%----------------------------------------------------------------------------------------

\section{Methodology and Approach}

To address these issues, a comprehensive literature review, including an analysis of white papers and existing literature in the area, will be conducted to benchmark the proposed solutions against the current state of the art. To address the different information needs of team roles, a novel solution for a logging decision management system will be designed, incorporating the concept of 'views'. This system will make use of knowledge graph technology to ensure the seamless integration and accessibility of decision logs and knowledge within the documentation practices of DevOps.

Data on the usability, effectiveness, and impact of the proposed solution on team collaboration and productivity will be collected through personalized interviews with DevOps professionals. This qualitative research will provide insights into the practical implications of the implemented solution, and contribute to the advancement of the state of the art in knowledge management and decision logging for DevOps teams.

%----------------------------------------------------------------------------------------




