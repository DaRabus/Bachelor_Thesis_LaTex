% Indicate the main file. Must go at the beginning of the file.
% !TEX root = ../main.tex

%----------------------------------------------------------------------------------------
% Chapter 1
%----------------------------------------------------------------------------------------



\chapter{Introduction} % Main chapter title
\label{Chapter1} % For referencing the chapter elsewhere, use \ref{Chapter1} 

%----------------------------------------------------------------------------------------

% Define some commands to keep the formatting separated from the content
% Placing such commands in the preamble is a good idea.
\newcommand{\keyword}[1]{\textbf{#1}}
\newcommand{\tabhead}[1]{\textbf{#1}}
\newcommand{\code}[1]{\texttt{#1}}
\newcommand{\file}[1]{\texttt{\bfseries#1}}
\newcommand{\option}[1]{\texttt{\itshape#1}}

%----------------------------------------------------------------------------------------



%----------------------------------------------------------------------------------------

\section{Background and Significance}

During my studies and professional career in software development, I have observed a critical challenge in the field of knowledge management within DevOps teams. This observation has motivated my interest in the implementation of structured decision-logging and knowledge management practices, with the aim of enhancing team efficiency and project success.

In 1990, Pinto's study \cite{Pinto1990} demonstrated that effective communication within and across teams was essential for the success of new program development. His research demonstrates the advantages of cross-functional team structures, where the transfer of knowledge between different functional areas leads to the generation of more innovative solutions and more efficient problem-solving. This is of particular relevance within the context of agile environments, such as those utilising SCRUM methodologies, where there is an inherent interdependence between roles, necessitating effective collaboration. Effective communication within teams leads to improved collaboration and, consequently, enhanced productivity. By ensuring the appropriate distribution of data and information among the team, projects are more likely to proceed according to plan, thereby achieving set objectives. Conversely, Humble and Molesky \cite{HumbleMolesky2011} discuss the benefits of DevOps (Developing and Operational practices) in enhancing communication by integrating project teams and operations. This integration ensures that all stakeholders are involved throughout the software delivery process. Additionally, the authors emphasise that DevOps is not merely a matter of technical practices but also encompasses cultural elements, automation, measurement, and sharing. Furthermore, they highlight the significance of knowledge and tools transfer across development and operations, which is essential for enhancing service delivery and risk management. The integration of diverse IT functions through DevOps fosters a collaborative decision-making environment, where shared knowledge and insights lead to informed and effective solutions. In order to address the challenges posed by complex systems and align IT outputs with the needs of the business, it is essential that a collaborative environment is created.

This Bachelor's thesis, titled "Defining a Structure for Decision Logging and Knowledge Management in DevOps Teams," endeavors to address the critical deficit in systematic decision logging and knowledge management practices that could exert a significant effect on the collaboration and productivity of these teams. As Mazure \cite{Mazur2023} notes, the effective utilisation of expertise and the implementation of effective solutions are dependent upon ensuring that team members are aligned and informed with regard to quality metrics and progress. This can be achieved through the use of modern tools and technologies to facilitate communication.
%----------------------------------------------------------------------------------------

\section{Research Questions}

This thesis aims to examine the documentation practices of a team and explore ways to integrate diverse methodologies to enhance productivity and collaboration. Three key research questions are addressed:

\begin{enumerate}[start=1,label={(\bfseries R\arabic*):}]
\item How to define the right format for documenting the right data for each team role in a SCRUM team?
\item How to implement a solution for maintaining and structure decision logs?
\item To what extent can collaboration and productivity be improved through these practices?
\end{enumerate}

\subsection{How to define the right format for documenting the right data for each team role in a SCRUM team?}
The objective of this question is to identify an optimal approach for documenting the activities and tasks in a SCRUM team. In order to achieve this, it is necessary to research what the relevant parts of a documentation are and how we can ensure that the people gain an overview without being overwhelmed by irrelevant information. Our objective is to create a dedicated set of information for each role in the team. Furthermore, the objective is to enhance the currency of the documentation while encouraging collaboration with stakeholders.

\subsection{How to implement a solution for maintaining and structure decision logs?}
A decision log is a set of information in a project which determines the choices of business and technical decisions over time. If this document becomes out of date, newly introduced team members or external parties may encounter difficulties in understanding certain decisions or the point of view of how to solve issues within the team or the project itself. The purpose of this question is to ascertain an appropriate method of storing and accessing information in a timely manner, ensuring it is available to the relevant individuals.

\subsection{To what extent can collaboration and productivity be improved through these practices?}
In order to bring this line of inquiry to a conclusion, this sub-question seeks to consolidate the findings yielded by the preceding questions. It draws together the insights gleaned from the information and lessons learned from the previous question and presents a synthesis of the potential outcomes that can be attained by implementing various approaches.
A key takeaway from this investigation is that the productivity and collaboration of a project team are crucial elements in the successful completion of a project.

%----------------------------------------------------------------------------------------

\section{Methodology and Approach}

This section provides an overview of the research methodology. Firstly, the chosen research method is discussed. This is followed by a description of the different stages of the research and the activities carried out at each stage.

The purpose of this study was to investigate the current state of art in SCRUM Teams of how the collaboration and the knowledge managment is working, where do Teams store their shared knowledge and how is it maintained, or is it maintained at all? As \cite{HumbleMolesky2011} is mentioning:

\begin{quote}
In most organizations, IT operations
consumes by far the majority of the IT budget. If you
can drive operating costs down by preventing and
removing bloat within systems created by projects,
you’d have more resources to focus on problem solving
and continuous improvement of IT services.
\end{quote}

Operations are quite expensive and consume a lot of money if handled ad-hoc and without the care. As \cite{Azad2023DevOps} is mentioning:

\begin{quote}
For software process integration the infrastructure play a crucial role
\end{quote}

Infrastructure is aswell as the knowledge a very important aspect within the Team, how can the infrastructure beeing managed and be deployed at the right place and within the right responsibilities.

The process model has been divided into three different stages.

\begin{enumerate}
    \item Problem identification
    \item Design and proposed solution
    \item Validation based on a Survey
\end{enumerate}


%----------------------------------------------------------------------------------------




