% Indicate the main file. Must go at the beginning of the file.
% !TEX root = ../main.tex

%----------------------------------------------------------------------------------------
% CHAPTER 2
%----------------------------------------------------------------------------------------

\chapter{Code Listings}

\label{Chapter2} % For referencing the chapter elsewhere, use \ref{Chapter2} 

%----------------------------------------------------------------------------------------

The package \href{https://www.overleaf.com/learn/latex/Code\_listing}{\code{listings}} permits to easily include existing code. Simply use the command \verb|\lstinputlisting[language=name]{path/to/file}|. See \href{https://www.overleaf.com/learn/latex/Code\_listing#Supported\_languages}{here} for a list of supported programming languages.


\lstinputlisting[language=Python,caption=External file: code/example.py]{Code/example.py}

It is also possible to enter code directly into \LaTeX:

\begin{lstlisting}[language=C++]
#include <stdio>
void hello_world(void){
   std::cout << "Hello World!" << std::endl;
}    
\end{lstlisting}

Alternatively, one can use the syntax highlighting toolbox \href{https://pygments.org/}{\code{Pygments}} in combination with the \LaTeX-package \href{www.overleaf.com/learn/latex/Code\_Highlighting\_with\_minted}{\code{minted}}. It provides slightly better results, as the code will actually be parsed.

To install Pygments, use the following command. For \code{minted} to work properly, run the pdflatex tool with the flag \code{--shell-escape}. If you are using a TEX editor, you can modify the typesetting   command somewhere in the settings.

\begin{lstlisting}[language=bash]
# Make sure that Pygments is installed.
python -m pip install pygments

# Then add the --shell-escape flag to the command 
# that is used to compile your LaTeX code.
pdflatex --output-dir="$BUILD_DIR" \
         --file-line-error \
         --shell-escape \
         --synctex=1 "$1"
\end{lstlisting}



% Set the following line to \iftrue if minted is available on your system.
% See the above instructions to see how.

\iffalse
See below how the result looks like if minted is available on your system.

\begin{listing}[!ht]
\inputminted[linenos, bgcolor=codebackground, style=friendly]{python}{Code/example.py}
\caption{Example from external file, parsed using \code{Pygments}}
\end{listing}

\fi