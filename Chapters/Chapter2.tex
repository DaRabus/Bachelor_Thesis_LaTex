% Indicate the main file. Must go at the beginning of the file.
% !TEX root = ../main.tex
%----------------------------------------------------------------------------------------
% CHAPTER 2
%----------------------------------------------------------------------------------------

\chapter{Current State of Art}

\label{Chapter2} % For referencing the chapter elsewhere, use \ref{Chapter2} 

%----------------------------------------------------------------------------------------
As Humble and Molesky \cite{HumbleMolesky2011} point out that the four most fundamental aspects are "culture", "automation", "measurement" and "sharing" we are going deeper into what these different aspects are in depth and how they can help us to tackle the evolving ecosystem. DevOps has increased the speed of the deployment and management of products as mentioned in \cite{Zhu2016DevOps}, this also introduces new challenges. In \cite{Zhu2016DevOps} is talking about the affects of cultural and technical aspects, which are associated within the transformation of a organization.
Within this evolving ecosystem, DevOps emerges as a critical element, merging software development (Dev) and IT operations (Ops) to streamline system development lifecycles and improve software quality.

However, Jabbari \cite{Jabbari2016} notes in his paper that DevOps is not merely about tools and processes; it is also about fostering a culture that enhances communication and collaboration among all stakeholders involved in software development and delivery. Current definitions focus on development and operations, but future iterations may include a broader integration with business operations. This would make DevOps a central part of business strategy rather than just a software development methodology. The term "DevOps" is open to a range of interpretations, encompassing toolchains, cultural practices, job titles, and responsibilities. Humble and Molesky identified four fundamental pillars of DevOps, as outlined in their 2011 paper, "The Four Pillars of DevOps" \cite{HumbleMolesky2011}.

\begin{itemize}
    \item \textbf{Culture:} The sharing of responsibility for the delivery of software
    \item \textbf{Automation:} Automation is a useful tool for increasing the efficiency of processes
    \item \textbf{Measurement:} The quantification of processes helps to understand capability and to set goals
    \item \textbf{Knowledge Sharing:} The most crucial aspect of knowledge sharing is the method of sharing
\end{itemize}


Of these, Knowledge Sharing is of central importance as it underpins the effective adoption of DevOps practices and facilitates knowledge management in the organization
\cite{Azad2023DevOps} .

\section{Culture}

The results presented in reference to Hermawan \cite{Hermawan2021DevOpsTeamwork} demonstrate that the integration of development and operational teams through the implementation of DevOps practices significantly enhances teamwork quality. The fostering of a collaborative culture, characterised by the encouragement of knowledge sharing and efficient teamwork, is a key outcome of DevOps. The continuous improvement and feedback cycles inherent to DevOps engage all team members in the iterative processes of development, leading to increased accountability and a more dynamic work environment. The implementation of the DevOps methodology ensures that team members provide mutual support and respect each other’s contributions, thereby facilitating collaboration in the effective management of challenges, ultimately leading to a more productive and harmonious work environment.

As previously noted by Mazur \cite{Mazur2023}, effective leadership is crucial in establishing and maintaining quality expectations within an organisation. By emphasising quality and demonstrating commitment to this goal, leaders can create an environment where employees are motivated to adhere to those standards, which subsequently enhances team collaboration and product outcomes. Implementing continuous improvement mechanisms and providing regular feedback within teams can facilitate improvements in working practices and product quality. Furthermore, teams that adopt a learning-oriented approach, where members engage in collective adaptation and knowledge sharing, tend to develop a stronger collaborative culture.

Furthermore, De Church and colleagues' research \cite{Dechurch2010} indicates that collaboration and effective communication between team members are significant factors in an organisation's success. It suggests that the benefits of team cognition vary depending on team and task characteristics. For example, highly interdependent teams may gain more from well-developed transactive memory systems than less interdependent teams. The research also highlights the importance of shared mental models in team effectiveness. These models assist team members in developing compatible understandings of their tasks and environment, thereby facilitating coordinated and efficient task performance. The study further indicates that transactive memory systems, which entail a shared understanding of which team members possess specific information, are vital for effective information processing and utilisation within teams. The findings indicate that the benefits of team cognition vary according to team and task characteristics. To illustrate, interdependent teams may derive greater benefits from well-developed transactive memory systems than less interdependent teams.

Pinto \cite{Pinto1990} defines the quality of communication within a team as the frequency and formalisation of the information exchange between team members. Frequency is determined by the frequency with which communication occurs among team members and the time spent on it. Formalisation refers to the degrees of spontaneity in the communication. Communication that requires significant planning and includes written status reports is considered formal, while spontaneous communication, such as talking in the doorway, chatting, or talking in front of a screen, is considered informal. It can be observed that ideas and contributions are typically shared, discussed, and evaluated with other team members in a timelier and more efficient manner through informal communication than formal communication. 


\section{Automation}

The article "Prioritizing Documentation Effort" \cite{McBurney2017PrioritizingDocEffort} outlines the potential of automation in the prioritization process of documentation efforts. This suggests that automation can assist in determining which sections of code require the most detailed documentation, thus optimizing the use of limited time and resources. The research findings indicate that textual analysis of source code, using techniques like Vector Space Models (VSM) and textual comparison, is effective in predicting documentation priorities. This implies that the implementation of artificial neural networks to automate the prioritisation process demonstrates the potential of machine learning technologies to enhance the efficiency and accuracy of determining documentation needs, with the capability of handling large codebases.

In his paper, Procko \cite{Procko2024CodeDocumentation} demonstrates the efficiency of automation through integration with Git repositories for real-time generation of code documentation. This eliminates the need for manual updates, significantly speeding up the documentation process. Procko's work also addresses the issue of 'documentation debt' – a situation whereby changes made to code are not reflected in its documentation. The automation of the documentation process ensures that the documentation remains consistent with the code, thereby reducing the likelihood of outdated information. Furthermore, the integration of the ACDC system into Continuous Integration and Continuous Delivery (CI/CD) pipelines demonstrates the potential for automation to streamline software development processes. This integration helps to ensure that up-to-date documentation is maintained without requiring additional effort from developers, thus allowing them to focus their attention on coding tasks rather than on documentation. The utilisation of Generative Pre-trained Transformers (GPT) and abstract syntax trees (ASTs) in the ACDC system exemplifies the role of advanced technologies in automating complex tasks. These technologies facilitate the generation of accurate, context-sensitive documentation in an automated manner. By automating the documentation generation process, the ACDC system allows developers to allocate more time and resources to other crucial aspects of software development, thereby enhancing overall productivity.
 

\section{Measurement}

The document 'A Performance Evaluation Model for DevOps Practices on Open Source Software Projects', written by Manuel Sánchez Ruiz \cite{Ruiz2023}, suggests that measuring aspects such as release frequency and lead time for changes enables teams to gauge their efficiency and productivity. The release frequency is utilized to assess how often a team successfully updates their project, which can be a direct indicator of the team’s ability to work effectively together and push updates regularly. The assessment of the lead time for changes allows the identification of the interval between code commit and code release. A reduction of this period may indicate the existence of a collaborative environment, characterised by the rapid development and deployment of changes. Conversely, a prolonged lead time may suggest a less efficient and collaborative team dynamic, where issues are not addressed promptly, thus affecting productivity and the health of the project.

Furthermore, Floris Erich \cite{Erich2017DevOps} notes that measurement is closely tied to performance improvements in DevOps practices. By measuring key performance indicators, organisations can better understand the contributions of team collaboration to overall productivity and quality of software delivery. However, aligning measurement practices with the dynamic and integrated nature of DevOps teams can be complex. Effective measurement strategies are therefore needed to accurately capture the nuances of collaboration within DevOps teams. The measurement of the frequency and effectiveness of interactions between team members can provide insights into the efficacy with which teams collaborate in the achievement of shared objectives.


\section{Knowledge Sharing}


The value of knowledge sharing in solving complex problems more effectively is emphasised by Crosby in his book \cite{Crosby2023}, The Business Manager's Guide to Software Projects, where he cites a number of examples to illustrate his point. By sharing diverse expertise and perspectives, teams can develop more comprehensive solutions that consider multiple aspects of a problem. Furthermore, knowledge sharing helps in standardising best practices across teams. This standardisation leads to more consistent and efficient processes, since team members are more likely to adopt proven methods rather than reinventing the wheel. Knowledge sharing among team members expedites the onboarding process for newcomers, while simultaneously aiding the advancement of skills amongst existing members. The continuous learning environment that ensues encourages a culture of continuous improvement and adaptability. With a shared knowledge base, team members have access to a broader array of information, which can lead to more informed decision-making. This is particularly pertinent in dynamic environments where decisions need to be made swiftly and efficiently. The breakdown of information silos, which knowledge sharing facilitates, engenders a more collaborative and integrated team environment. This integration serves to align the team's efforts towards a unified goal, thereby enhancing overall team performance.


As evidenced by Kuusinen \cite{Kuusinen2017} in her survey of the subject, the results demonstrate that the application of more agile techniques is correlated with increased ease of sharing knowledge among team members. Agile environments facilitate open communication and continuous feedback, both of which are essential for effective collaboration in the resolution of problems. However, while the implementation of agile methods has been demonstrated to significantly enhance the sharing of knowledge within teams, they have not proven to be as efficacious in facilitating knowledge transfer between teams or across organisational boundaries. These results imply the necessity for organisational strategies to promote the dissemination of knowledge beyond the immediate team environment. The study highlights the distinction in motivation between knowledge sharing within teams and knowledge sharing with external parties (colleagues within the organisation or customers). Within teams, intrinsic motivations (e.g. enjoyment of collaborative work) prevail, whereas extrinsic factors (e.g. building customer relationships) play a more significant role in interactions with external stakeholders. An understanding of these motivational nuances can assist in the development of tailored approaches to improve knowledge sharing practices in a more effective manner. The survey indicated that informal communication methods, such as casual discussions and the use of whiteboard technology, were more effective for the sharing of knowledge within teams than formal methods, including presentations and emailing. This emphasises the significance of fostering an environment that encourages spontaneous and informal interactions among those who work together. The study indicates that the sharing of knowledge beyond the immediate team remains challenging, primarily as a consequence of a lack of organisational support and the absence of appropriate incentives. The development of a culture that values the sharing of knowledge throughout the organisation is of critical importance to the improvement of collaboration and efficiency on a larger scale.