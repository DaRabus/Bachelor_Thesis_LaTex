% Indicate the main file. Must go at the beginning of the file.
% !TEX root = ../main.tex
%----------------------------------------------------------------------------------------
% CHAPTER 2
%----------------------------------------------------------------------------------------

\chapter{Current State of Art}

\label{Chapter2} % For referencing the chapter elsewhere, use \ref{Chapter2} 

%----------------------------------------------------------------------------------------
As Humble and Molesky \cite{HumbleMolesky2011} point out that the four most fundamental aspects are "culture", "automation", "measurement" and "sharing" we are going deeper into what these different aspects are in depth and how they can help us to tackle the evolving ecosystem. DevOps has increased the speed of the deployment and management of products as mentioned in \cite{Zhu2016DevOps}, this also introduces new challenges. In \cite{Zhu2016DevOps} is talking about the affects of cultural and technical aspects, which are associated within the transformation of a organization.
Within this evolving ecosystem, DevOps emerges as a critical element, merging software development (Dev) and IT operations (Ops) to streamline system development lifecycles and improve software quality.

However, despite its growing adoption, the concept of DevOps lacks a consistent academic definition \cite{jabbari2016devops}. It encompasses a range of interpretations, from toolchains and cultural practices to job titles and responsibilities. Humble and Molesky identified four fundamental pillars of DevOps \cite{HumbleMolesky2011}:

\begin{itemize}
    \item \textbf{Culture:} The sharing of responsibility for the delivery of software
    \item \textbf{Automation:} Automation is a useful tool for increasing the efficiency of processes
    \item \textbf{Measurement:} The quantification of processes helps to understand capability and to set goals
    \item \textbf{Knowledge Sharing:} The most crucial aspect of knowledge sharing is the method of sharing
\end{itemize}


Of these, Knowledge Sharing is of central importance as it underpins the effective adoption of DevOps practices and facilitates knowledge management in the organization. 
\cite{Azad2023DevOps}

\section{Culture}

The results presented in reference to Hermawan \cite{Hermawan2021DevOpsTeamwork} demonstrate that the integration of development and operational teams through the implementation of DevOps practices significantly enhances teamwork quality. The fostering of a collaborative culture, characterised by the encouragement of knowledge sharing and efficient teamwork, is a key outcome of DevOps. The continuous improvement and feedback cycles inherent to DevOps engage all team members in the iterative processes of development, leading to increased accountability and a more dynamic work environment. The implementation of the DevOps methodology ensures that team members provide mutual support and respect each other’s contributions, thereby facilitating collaboration in the effective management of challenges, ultimately leading to a more productive and harmonious work environment.

As previously noted by Mazur \cite{Mazur2023ProblemProductQuality}, effective leadership is crucial in establishing and maintaining quality expectations within an organisation. By emphasising quality and demonstrating commitment to this goal, leaders can create an environment where employees are motivated to adhere to those standards, which subsequently enhances team collaboration and product outcomes. Implementing continuous improvement mechanisms and providing regular feedback within teams can facilitate improvements in working practices and product quality. Furthermore, teams that adopt a learning-oriented approach, where members engage in collective adaptation and knowledge sharing, tend to develop a stronger collaborative culture.

Furthermore, Dechurch and colleagues' research \cite{Dechurch2010} indicates that collaboration and effective communication between team members are significant factors in an organisation's success. It suggests that the benefits of team cognition vary depending on team and task characteristics. For example, highly interdependent teams may gain more from well-developed transactive memory systems than less interdependent teams. The research also highlights the importance of shared mental models in team effectiveness. These models assist team members in developing compatible understandings of their tasks and environment, thereby facilitating coordinated and efficient task performance. The study further indicates that transactive memory systems, which entail a shared understanding of which team members possess specific information, are vital for effective information processing and utilisation within teams. The findings indicate that the benefits of team cognition vary according to team and task characteristics. To illustrate, interdependent teams may derive greater benefits from well-developed transactive memory systems than less interdependent teams.

Pinto \cite{Pinto1990} defines the quality of communication within a team as the frequency and formalisation of the information exchange between team members. Frequency is determined by the frequency with which communication occurs among team members and the time spent on it. Formalisation refers to the degrees of spontaneity in the communication. Communication that requires significant planning and includes written status reports is considered formal, while spontaneous communication, such as talking in the doorway, chatting, or talking in front of a screen, is considered informal. It can be observed that ideas and contributions are typically shared, discussed, and evaluated with other team members in a timelier and more efficient manner through informal communication than formal communication. 


\section{Automation}

The article "Prioritizing Documentation Effort" \cite{McBurney2017PrioritizingDocEffort} outlines the potential of automation in the prioritization process of documentation efforts. This suggests that automation can assist in determining which sections of code require the most detailed documentation, thus optimizing the use of limited time and resources. The research findings indicate that textual analysis of source code, using techniques like Vector Space Models (VSM) and textual comparison, is effective in predicting documentation priorities. This implies that the implementation of artificial neural networks to automate the prioritisation process demonstrates the potential of machine learning technologies to enhance the efficiency and accuracy of determining documentation needs, with the capability of handling large codebases.

In his paper, Procko \cite{Procko2024CodeDocumentation} demonstrates the efficiency of automation through integration with Git repositories for real-time generation of code documentation. This eliminates the need for manual updates, significantly speeding up the documentation process. Procko's work also addresses the issue of 'documentation debt' – a situation whereby changes made to code are not reflected in its documentation. The automation of the documentation process ensures that the documentation remains consistent with the code, thereby reducing the likelihood of outdated information. Furthermore, the integration of the ACDC system into Continuous Integration and Continuous Delivery (CI/CD) pipelines demonstrates the potential for automation to streamline software development processes. This integration helps to ensure that up-to-date documentation is maintained without requiring additional effort from developers, thus allowing them to focus their attention on coding tasks rather than on documentation. The utilisation of Generative Pre-trained Transformers (GPT) and abstract syntax trees (ASTs) in the ACDC system exemplifies the role of advanced technologies in automating complex tasks. These technologies facilitate the generation of accurate, context-sensitive documentation in an automated manner. By automating the documentation generation process, the ACDC system allows developers to allocate more time and resources to other crucial aspects of software development, thereby enhancing overall productivity.
 

\section{Measurement}

The quantitative assessment of DevOps practices within SCRUM teams, particularly in the context of process improvement, necessitates a meticulous approach to measurement. This section delves into the application of performance metrics adapted from open-source software (OSS) projects to the SCRUM framework, highlighting the significance of these metrics in enhancing team collaboration and productivity.

The pioneering performance evaluation model for DevOps practices in OSS projects presents a valuable framework for SCRUM teams. By adapting key metrics from this model—such as Release Frequency, Lead Time for Released Changes, Time to Repair Code, and Bug Issues Rate—this section explores their relevance and application in measuring and improving the SCRUM process.\cite{PerformanceEvalOSS2023}

\textbf{Release Frequency} and \textbf{Lead Time for Released Changes} serve as crucial indicators of a team's agility and responsiveness. The iterative nature of SCRUM, which emphasizes quick cycles of development and feedback, aligns well with these metrics, offering insights into the efficiency of integrating and deploying new functionalities or corrections.\cite{PerformanceEvalOSS2023}

\textbf{Time to Repair Code} and \textbf{Bug Issues Rate} are employed to assess the quality and stability of the team's outputs. Within the dynamic environment of a SCRUM team, these metrics are instrumental in monitoring the maintenance of high-quality standards amidst rapid development phases.\cite{PerformanceEvalOSS2023}

Drawing inspiration from the automated, metric-based Performance-Tracker tool designed for OSS projects, this subsection outlines a structured methodology to quantitatively analyze SCRUM processes. This approach leverages automated data collection from project management tools to capture the adapted performance metrics, thus facilitating a detailed assessment of SCRUM team efficiency and productivity.\cite{PerformanceEvalOSS2023}

\begin{enumerate}
    \item \textbf{Automated Data Collection:} Implementing automated tools to extract data on key performance metrics directly from SCRUM project management platforms.
    \item \textbf{Analytical Evaluation:} Calculating the average and standard deviation for each performance metric to understand the stability and performance trends of the SCRUM team over time.
    \item \textbf{Performance Benchmarking:} Conducting a comparative analysis with established benchmarks or historical team performance to identify improvement areas and best practices.
\end{enumerate}


\section{Knowledge Sharing}

For the success and continuous improvement of software delivery and operations processes, knowledge sharing within DevOps teams is essential. Crosby's work highlights the importance of creating an environment in which decision making and team collaboration are not only encouraged, but are a fundamental part of the project framework \cite{Crosby2023}. This environment is in line with the DevOps pillar of culture, which advocates the sharing of responsibility for software delivery. It emphasises the role of effective communication and documentation practices in facilitating knowledge transfer, thereby improving team dynamics and project outcomes.

The adoption of SCRUM within DevOps practices plays a critical role in structuring the flow of knowledge. Crosby suggests that decision making and collaboration frameworks used in business can be adapted to SCRUM teams to increase their efficiency and productivity \cite{Crosby2023}. This approach aligns with the DevOps pillar of knowledge sharing. SCRUM methodologies inherently promote transparency, regular feedback and continuous learning. By defining the right format for documenting data that is tailored to the role of each team member, SCRUM teams can ensure that critical knowledge is accessible, understandable, and actionable.

As highlighted by Crosby \cite{Crosby2023}, improving knowledge sharing in DevOps teams requires a systematic approach to documenting and visualising decision logs. As well as supporting the automation and measurement pillars of DevOps, this approach reinforces the foundation for a culture of openness and continuous improvement. Team members can be empowered to make informed decisions, learn from past experiences and set clear goals for future development by implementing tools and practices that provide easy access to documented decisions and performance metrics.

\section{Conclusion}

A multifaceted approach essential to advancing software development and operations practices has emerged from examining DevOps through the lenses of culture, automation, measurement, and knowledge sharing. This chapter highlighted how these four pillars transform the software development lifecycle, resulting in more efficient, reliable, and collaborative processes.

\textbf{Culture} has emerged as the foundation of DevOps, advocating shared responsibility in the process of delivering software. The success and maintainability of software projects is ensured through the adoption of DevOps and SCRUM methodologies, which foster a collaborative environment that improves both product quality and documentation.

\textbf{Automation} represents a significant leap forward in streamlining software development workflows, facilitated by advances in AI and CI/CD practices. Tools such as GPT and the ACDC system not only improve productivity and documentation quality. They also promote a culture of continuous improvement and knowledge sharing.

\textbf{Measurement} plays a critical role in the refinement and optimisation of DevOps practices. Adopting quantitative measurement techniques used in OSS projects and using these within a SCRUM framework allows teams to gain valuable insight into their operations, identify areas for improvement, and improve productivity and cooperation.

\textbf{Knowledge Sharing} has been identified as a key element in the achievement of operational excellence and innovation. Based on Crosby's insights, integrating systematic documentation and visualisation techniques into DevOps teams significantly increases collaboration, productivity and project success.

In conclusion, a comprehensive framework for overcoming traditional challenges in software development is provided by the confluence of culture, automation, measurement and knowledge sharing in DevOps practices. The importance of each pillar in building a resilient, adaptive and high-performing software development environment has been highlighted throughout this chapter. As DevOps evolves, these pillars will be the guiding principles for organisations striving for excellence in software development and operations.

%// Go deeper into these topics and deeper what is exactly missing for the gap 