% Indicate the main file. Must go at the beginning of the file.
% !TEX root = ../main.tex

%----------------------------------------------------------------------------------------
% CHAPTER 3
%----------------------------------------------------------------------------------------

\chapter{Design and proposed solution}

\label{Chapter3} % For referencing the chapter elsewhere, use \ref{Chapter2} 

%----------------------------------------------------------------------------------------
In the previous chapter, we explored four of the key pillars of DevOps and concluded that the related issues for solving the key problems within a SCRUM team, and in DevOps in general, are not a single point of truth, but rather a whole web of aspects to be solved. The aim of this chapter is to propose a solution that addresses the challenges identified. The focus is on efficiently managing and accessing information tailored to the different roles within a SCRUM team.


\section{Requirements and Features}

% Find more requirements and features and propose a survey about it, so we can generate a validation survey if it makes sense to 
% https:%aisel.aisnet.org/cgi/viewcontent.cgi?article=1098&context=ecis2003

Designing a successful solution is based on the principle that the right information needs to be available in a timely manner, tailored to the specific needs of each role within a SCRUM team. In order to do this, the design of the solution must have the following key features

\begin{itemize}
    \item \textbf{Role-based Information Highlighting:} The system should allow users to access the information most relevant to their responsibilities by selecting their role (e.g. business analyst, project manager, developer). This will ensure that team members are not overwhelmed with unnecessary detail. They will be able to focus on the information that is relevant to their role. This process should be maintained and ensured by an AI Model which is configured for the right roles and is trying to highlight the information based on keywords used in the Text.
    \item \textbf{Comprehensive View Option:} While role-specific information is critical, the ability to view the documentation as a whole ("all roles") is equally important for a holistic understanding of the project and for roles that require a broader overview.
    \item \textbf{Collaborative Documentation Process:} The creation of documentation needs to involve the input of all members of the team to ensure that the information needs of the different roles are adequately represented and addressed. This cooperative approach promotes a more inclusive and comprehensive knowledge management system.
    \item \textbf{Automation to improve the maintainability} The maintenance of the documentations should be
    possible to be semi automated within AI Toolings, a possible implementation could be determinated by the documentation in the Code Base which is then abstracted within AI Toolings to the right papers.
    \item \textbf{Maintainability of the Decision Logs within the Mirrored Approach} The aim of the decision logs is to ensure that they are always maintained and also mirrored, both to the code base they refer to and to documentation important to business and "architectural" decisions.
    \item \textbf{Infrastructure-as-Code to sustain shared responsibility} Getting everyone in the development team on the same page is the goal of Infrastructure as Code with tools like Terraform. By automating the CI / CD process, the code that goes into the 'main' branch of the project is checked within deployments and tests that are all controlled by Infrastructure as Code, and the pods are also recreated for E2E testing, ensuring a real test environment that is very close to the production environment.
\end{itemize}

% https:%devops.com/mastering-the-shared-responsibility-model/ 
% Infrastructure as code as 

\section{Identification of SCRUM Team Roles}

A SCRUM team has several key roles, each with different responsibilities:

% Add Personas to the roles
% https:%www.interaction-design.org/literature/article/personas-why-and-how-you-should-use-them

\begin{itemize}
    \item \textbf{Product Owner:} Responsible for defining the vision of the project, managing the product backlog, and ensuring that the team is delivering value to the business.
    \item \textbf{Scrum Master:} Facilitating SCRUM processes, helping the team to remove barriers and ensure efficient workflows.
    \item \textbf{Development Team:} Includes front-end and back-end developers focusing on different aspects of the system, but working closely together. Developers will need to have access to role-specific technical documentation and integration guidelines.
    \item \textbf{Quality Assurance (QA) Engineers:} Focusing on testing, ensuring that the software meets every need or standard. Require detailed project specifications and test logs.
    \item \textbf{Business Analysts (BAs):} Translate business requirements into technical specifications and bridge the gap between IT and the business. Both business requirements and technical implementations need to be fully understood by BAs.
\end{itemize}


\section{Personas for SCRUM Team Roles}

To tailor our proposed solution to the diverse needs within a SCRUM team, we introduce detailed personas representing each key role. These personas are designed to guide the development process by highlighting the specific requirements and backgrounds of team members.

\subsection*{Persona: Emma - Product Owner}
\begin{description}
    \item[Background:] Emma has a rich background in product management, with extensive experience in the technology sector. She is adept at defining the vision of projects and managing product backlogs.
    \item[Needs:] She needs to have a high-level overview of project progress and clear, concise summaries of technical decisions and their business implications.
    \item[Information Needs:] In daily SCRUMs and planning meetings, Emma seeks concise updates on progress towards goals, impediments to the backlog, and insights into user feedback. Her focus is on ensuring that the team's efforts align with strategic business objectives and on making informed prioritization decisions.
\end{description}

\subsection*{Persona: Lucas - Scrum Master}
\begin{description}
    \item[Background:] Lucas is a certified SCRUM master who champions agile methodologies. He plays a pivotal role in facilitating SCRUM processes and helping the team remove barriers to ensure efficient workflows.
    \item[Needs:] Lucas requires tools and documentation that aid in workflow management, team performance tracking, and identification and resolution of impediments.
    \item[Information Needs:] Lucas requires real-time updates on team progress, blockages, and the effectiveness of SCRUM ceremonies. His role involves identifying and addressing impediments, fostering team cohesion, and facilitating communication between the development team and stakeholders.

\end{description}

\subsection*{Persona: Mia - Front-End Developer}
\begin{description}
    \item[Background:] Specializing in front-end development, Mia works closely with designers to implement engaging user interfaces, using technologies like HTML, CSS, and JavaScript.
    \item[Needs:] She seeks detailed documentation on UI/UX design principles, integration with back-end systems, and project's visual and functional requirements.
    \item[Information Needs:] Mia looks for detailed design specifications, user experience feedback, and technical dependencies in daily updates. She values clarity on sprint tasks, visual guidelines, and any changes to user stories or priorities that affect the front-end development.
\end{description}

\subsection*{Persona: Alex - Back-End Developer}
\begin{description}
    \item[Background:] Alex focuses on server-side development, managing databases, and developing application logic. He is proficient in various programming languages and back-end frameworks.
    \item[Needs:] Alex looks for in-depth technical documentation on APIs, database schemas, system architecture, and performance optimization guidelines.
    \item[Information Needs:] In SCRUM meetings, Alex expects comprehensive information on API specifications, data models, and integration points with front-end features. He needs updates on back-end dependencies, system performance metrics, and security requirements to ensure robust and scalable solutions.
\end{description}

\subsection*{Persona: Sophia - QA Engineer}
\begin{description}
    \item[Background:] Sophia is dedicated to quality assurance, with a meticulous approach to testing and ensuring that software meets all standards and requirements.
    \item[Needs:] She requires access to comprehensive test plans, project specifications, detailed bug reports, and tools that support test automation and collaboration.
    \item[Information Needs:] She seeks detailed test plans, bug reports, and feature requirements during SCRUM meetings. Sophia emphasizes the importance of understanding new functionalities, regression areas, and the impact of changes to maintain high-quality standards across iterations.
\end{description}

\subsection*{Persona: Noah - Business Analyst}
\begin{description}
    \item[Background:] Noah bridges the gap between IT and business stakeholders, translating complex business requirements into technical specifications.
    \item[Needs:] He needs thorough documentation that clearly outlines project goals, requirements, and the technical implementations, ensuring a comprehensive understanding for both technical and non-technical stakeholders.
    \item[Information Needs:] Noah requires access to business goals, user insights, and technical constraints. In SCRUM ceremonies, he focuses on aligning the team's output with strategic objectives, clarifying requirements, and facilitating stakeholder feedback loops.
\end{description}

\subsection*{Persona: Mia - Front-End Developer}
\begin{description}
    \item[Background:] Mia is a creative and skilled Front-End Developer with a strong foundation in modern web development technologies including HTML5, CSS3, JavaScript, and frameworks like React.js. With her keen eye for design and user experience, Mia is adept at translating design concepts into dynamic and responsive web applications. She is enthusiastic about joining the SCRUM team and contributing to creating user-friendly interfaces that enhance user engagement.
    \item[Needs:] Mia requires up-to-date and comprehensive documentation on the project's UI/UX design guidelines, API integration protocols, and front-end development standards. As a new team member, she is particularly keen on accessing detailed project overviews and previous decision logs to better understand the project's evolution and rationale behind key architectural decisions.
    \item[Information Needs:] She looks forward to clear, concise, and regularly updated documentation that covers the spectrum of design principles, coding standards, and component libraries used in the project. Mia also values insights into pending updates or changes in front-end development practices within the team to align her work with the project's goals and timelines effectively.
\end{description}

\subsection*{Persona: Alex - Back-End Developer}
\begin{description}
    \item[Background:] Alex is a solution-oriented Back-End Developer with extensive experience in server-side logic, database management, and cloud-based infrastructure. His expertise encompasses a range of back-end technologies and languages, including Node.js, Python, and databases like PostgreSQL and MongoDB. Alex is passionate about building scalable and efficient back-end systems that support complex web applications. He is excited about the opportunity to collaborate with the SCRUM team to enhance the project's back-end capabilities.
    \item[Needs:] Alex seeks detailed and current documentation on the project's back-end architecture, database schemas, RESTful API specifications, and integration points with the front-end. He emphasizes the importance of understanding the security protocols, data handling practices, and performance optimization strategies implemented in the project.
    \item[Information Needs:] As he integrates into the team, Alex requires access to a centralized knowledge base where he can find information on the system's architecture, coding guidelines, and key decisions affecting the back-end development. He is also interested in regular updates on any new technologies or frameworks being adopted by the team to ensure that his contributions are in line with the latest back-end development trends and best practices.
\end{description}



\section{Situation Overview}

Facing the challenge of outdated documentation and decision logs, a SCRUM team comprising distinct personas seeks an innovative solution. This chapter outlines a strategy that leverages the four pillars of DevOps—Culture, Automation, Measurement, and Knowledge Sharing—to address these challenges, especially with the onboarding of two new developers: Mia, a Front-End Developer, and Alex, a Back-End Developer.

The team's current documentation and decision logs are significantly outdated. This poses a challenge for integrating Mia and Alex into the workflow efficiently. Concerns are raised by both business stakeholders and existing team members about the potential for information gaps. The team realizes the need for an updated approach to maintain their documentation and logs to ensure a smooth transition for the newcomers.

\section{Strategic Approach}
To navigate these challenges, the team decides to implement a solution that embodies the core principles of DevOps, tailored to their unique dynamics and incorporating the personas of Emma (Product Owner), Lucas (Scrum Master), Sophia (QA Engineer), and Noah (Business Analyst), along with Mia and Alex.

\subsection{Cultivating a Unified Culture}
The team, led by Emma and Lucas, acknowledges the importance of fostering a culture that values continuous improvement and collective responsibility for documentation. Regular team meetings are instituted to discuss updates, ensuring that all members, including Mia and Alex, are aligned and engaged in maintaining the project's documentation and decision logs.

\subsection{Facilitating Knowledge Sharing with Confluence}
To bolster knowledge sharing, Confluence as a centralized knowledge management platform is adopted. This platform serves as a repository for all documentation and decision logs, offering features like tagging and searching to facilitate easy access. The team commits to documenting their workflows and decision-making processes, thereby capturing and sharing tacit knowledge among team members, including the perspectives of Mia and Alex.

\subsection{Leveraging Automation with AI}
To solve the problem of outdated documentation, the team decides to use AI tools such as ChatGPT. This requires more effort to update the structure of the documentation and then adapt the automation to the capabilities, keeping decision logs in the relevant codebase and automatically synchronising them with the associated Confluence page. Code-based documentation should be considered as being interpreted by the AI rather than directly reflected in the documentation, this requires additional definitions for future development and involving everyone in the process of what exactly should be commented on to ensure the documentation is kept up to date with the relevant information.

\subsection{Implementing Infrastructure as Code iterative}
To ensure better onboarding in the future and to increase the level of shared responsibility across the project team, the team decides to implement Terraform as an Infrastructure-as-Code provider. The migration will not be possible immediately due to the increased effort within the overall structure, but for future iterations the migration will be planned and implemented iteratively within the next sprints. The IaC also helps to ensure that new team members have the opportunity to learn how the deployment and infrastructure work together, and the integration of the IaC helps to improve the maintenance of decision logs within the correct codebase, as with Terraform the deployments will be available as scripts.

\subsection{Onboarding Mia and Alex}
Introducing Mia and Alex has been carefully planned. The new changes may not apply to everything yet. But they are in place for the future. The new developers can already get involved in the new integration, as well as the new plans for the Terraform integrations. It's important that the new developers are made aware of this process as part of the new implementation for interpreting code comments in the right place in the documentation, so that they use it from the start. In addition, regular meetings will be scheduled to answer any questions they may have and to incorporate their feedback into the continuous improvement of the documentation process.

\section{Conclusion}
This proposed solution, centred on the DevOps pillars, presents a solution that involves additional work and new process integrations.To ensure resilient integration of these new processes, the entire process must be tracked within regular SCRUM sprints. Depending on the workload and the team's capacity for additional improvement stories, the new integrations may not be fully integrated before the new onboarding. Implementing this solution should make a significant contribution to the team's efficiency, productivity and overall success.
