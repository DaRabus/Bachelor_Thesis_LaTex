% !TEX root = ../main.tex

%----------------------------------------------------------------------------------------
% ABSTRACT PAGE
%----------------------------------------------------------------------------------------
\begin{abstract}
\addchaptertocentry{\abstractname} % Add the abstract to the table of contents
This Bachelor's thesis, addresses the critical need for systematic decision logging and effective knowledge management within \ac{DevOps} teams to enhance collaboration and productivity.

The study begins by exploring the background and significance of these practices, drawing from Pinto's (1990) research on the importance of communication in software development and the benefits of cross-functional teams in agile environments. The research questions focus on defining the optimal format for documenting data for different team roles in a \ac{SCRUM} team, implementing a structured solution for maintaining decision logs, and evaluating the extent to which these practices improve collaboration and productivity.

A comprehensive research methodology is employed, beginning with a literature review, followed by the development of personas to simulate team dynamics and identify gaps in current practices. These approaches are then validated based on a conducted survey within \ac{IT} professionals which are working in \ac{SCRUM} teams. Subsequently, tools and methodologies are examined to address these gaps, culminating in the creation of a practical blueprint for structured decision logging and knowledge management to improve collaboration and productivity.

The findings highlight the importance of role-based information highlighting, consistent and accessible decision logs, and the integration of \ac{IaC} tools to improve team collaboration and efficiency. The thesis concludes with a discussion on the potential outcomes of these practices and future research directions to further enhance \ac{DevOps} team performance.

\end{abstract}


%----------------------------------------------------------------------------------------
% German ABSTRACT PAGE
%----------------------------------------------------------------------------------------
\begin{extraAbstract}
\addchaptertocentry{\extraabstractname} % Add the abstract to the table of contents
Diese Bachelorarbeit befasst sich mit dem dringenden Bedarf an einer systematischen Entscheidungsprotokollierung sowie einem effektiven Wissensmanagement innerhalb von \ac{DevOps}-Teams, um die Zusammenarbeit und Produktivität zu verbessern.

Die Studie beginnt mit einer Untersuchung des Hintergrunds und der Bedeutung dieser Praktiken, wobei sie sich auf Pintos (1990) Forschung zur Bedeutung der Kommunikation in der Softwareentwicklung und den Vorteilen funktionsübergreifender Teams in agilen Umgebungen stützt. Im Rahmen der vorliegenden Studie werden zunächst die Forschungsfragen formuliert, welche sich mit der Definition des optimalen Formats zur Dokumentation von Daten für verschiedene Teamrollen in einem \ac{SCRUM}-Team, der Implementierung einer strukturierten Lösung zur Führung von Entscheidungsprotokollen sowie der Bewertung des Ausmasses beschäftigen, in dem diese Praktiken die Zusammenarbeit und Produktivität verbessern.

Im Rahmen dieser Forschungsarbeit wird eine umfassende Forschungsmethodik angewendet. Zu Beginn erfolgt eine Literaturrecherche. Anschliessend werden sogenannte Personas entwickelt, die die Dynamik innerhalb eines Teams nachbilden und Lücken in den aktuellen Praktiken identifizieren. Die Ergebnisse dieser beiden Schritte werden anhand einer empirischen Untersuchung mithilfe einer Umfrage validiert, welche unter \ac{IT}-Fachleuten in \ac{SCRUM}-Teams durchgeführt wurde. Im Anschluss werden Werkzeuge und Methoden untersucht, um die identifizierten Lücken zu schliessen.

Dies mündet in der Erstellung eines praktischen Leitfadens für strukturierte Entscheidungsprotokollierung und Wissensmanagement zur Verbesserung der Zusammenarbeit und Produktivität. Die Ergebnisse unterstreichen die Bedeutung der rollenspezifischen Hervorhebung von Informationen, konsistenter und zugänglicher Entscheidungsprotokolle sowie der Integration von \ac{IaC}-Werkzeugen zur Verbesserung der Teamzusammenarbeit und Effizienz. Die Arbeit endet mit einer Diskussion der potenziellen Ergebnisse dieser Praktiken und möglicher Forschungsrichtungen für eine weitere Verbesserung der \ac{DevOps}-Teamleistung.
\end{extraAbstract}
